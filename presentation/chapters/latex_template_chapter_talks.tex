\section{TeX Quickstart}
{%
\setbeamertemplate{frametitle}[chapter title]
\begin{frame}[fragile]
\frametitle<presentation>{TeX Quickstart}


\mode<article>{
\includethumb
}


\mode
<presentation>

\note{Mal gucken, wo das hier dann aufschlägt und wie das alles aussieht}

\mode
<article>

Das hier ist eine kurze Übersicht über alle möglichen TeX-Strukturen in einer B1-Unterlage. Für den Standardaufbau einer B1-TeX-Datei bitte die Templatedatei verwenden, die beim Anlegen neuer Projekte automatisch spendiert wurde. Abweichungen von dieser Struktur werden nicht akzeptiert.

Theoretische Hintergründe, mehr Details, Sprachspezifisches und Formulierungshilfen stehen im Styleguide, \url{https://dav.intern.b1-systems.de/documentation/trainings/internal/de/style-handout\_internal.pdf}.

\mode
<all>

\end{frame}
}
%%%%%%%%%%%%%%%%%%%%%%%%%%%%%%%%%%%%%%%%%%%%%%%%

\newpage

%%%%%%%%%%%%%%%%%%%%%%%%%%%%%%%%%%%%%%%%%%%%%%%%
\subsection{Allgemeine Konventionen}
%%%%%%%%%%%%%%%%%%%%%%%%%%%%%%%%%%%%%%%%%%%%%%%%
\begin{frame}[fragile]
\frametitle<presentation>{Allgemeine Konventionen}

\mode<article>{
\includethumb
}

\mode
<presentation>
\small
\begin{block}{\centering{Typografische Konventionen}}
\begin{center}
\begin{tabularx}{\textwidth}{lX}
\textbf{Element} & \textbf{Auszeichnung}\\
Datei- und Verzeichnisname & \verb+\texttt{/etc/passwd}+\\
Username & \verb+\texttt{tux}+ \\
Eingabe & \verb+\texttt{www.example.com}+ \\
URL & \verb+\url{http://www.suse.de}+ \\
Umgebungsvariablen & \verb+\texttt{PATH}+ \\
Platzhalter/Variablen & \verb+\texttt{\textit{filename}}+ \\
Kommandos & \verb+\texttt{rm -rf *}+ \\
Menüpunkt/Eingabefelder/Buttons & \verb+\textit{OK}+ \\
Tasten und -kombinationen & \verb+\textit{Alt+F1}+ \\
Manpage & Manpage (\verb+\texttt{man 1 man}+)\\
Einheiten & 5\verb+~+Gigabyte\\
\end{tabularx}
\end{center}
\end{block}
\normalsize


\mode
<article>
Die absoluten Basics an typografischen Auszeichnungen für unsere Materialien:

\begin{tabularx}{\textwidth}{|l|X|}
\hline
\textsc{\textbf{Element}} & \textsc{\textbf{Auszeichnung}}\\
\hline
\hline
Datei und Verzeichnisname & \textbackslash texttt\{/etc/passwd\}\\
Username & \textbackslash texttt\{tux\} \\
Eingabe & \textbackslash texttt\{www.example.com\} \\
URL & \textbackslash url\{http://www.b1-systems.de\} \\
Umgebungsvariablen & \textbackslash texttt\{PATH\} \\
Platzhalter/Variablen & \textbackslash texttt\{\textbackslash textit\{filename\}\} \\
Kommandos & \textbackslash texttt\{rm -rf *\} \\
Menübezeichnungen/Eingabefelder/Buttons & \textbackslash textit\{OK\} \\
Tasten und -kombinationen & \textbackslash textit\{Alt+F1\} \\
Manpage & Manpage (\textbackslash texttt\{man 1 man\})\\
Einheiten & 5$\sim$Gigabyte \\
\hline
\end{tabularx}


\mode
<all>

\end{frame}
\newpage

%%%%%%%%%%%%%%%%%%%%%%%%%%%%%%%%%%%%%%%%%%%%%%%%
\subsection{Tipps und Tricks}
%%%%%%%%%%%%%%%%%%%%%%%%%%%%%%%%%%%%%%%%%%%%%%%%
\begin{frame}
\frametitle<presentation>{Tipps und Tricks, I}

\mode<article>{
\includethumb
}

\mode
<presentation>
\begin{itemize}
\item Inhalt passt nicht auf die Folie? Notfalls Font verkleinern (small, footnotesize, scriptsize, tiny) und danach wieder auf normale Größe umschalten. Tipp: niemals kleiner als footnotesize auf Folien!
\note[item]{wen interessiert schon Layout? Los, sagt schon, wen?}
\item TeX ignoriert einfachen Zeilenumbruch; wer Umbruch will, mache zwei Umbrüche bzw. eine Leerzeile
\item ein Gedankenstrich entsteht aus zwei \texttt{-}
\item -{}- liefert die Doppelbindestriche, wie sie manchmal bei Befehlsangaben verwendet werden
\end{itemize}


\mode
<article>

\mode
<all>

\end{frame}
\newpage

\begin{frame}
\frametitle<presentation>{Tipps und Tricks, II}

\mode<article>{
\includethumb
}

\mode
<presentation>
\begin{itemize}
\item Bestimmte Zeichen haben eine eigene Bedeutung in TeX und müssen per Backslash maskiert werden. Lieblingskandidaten:
\begin{description}
  \item[\&] wird in Tabellen als Trenner verwendet
  \item[\%] Kommentarzeichen
  \item[\$] umschließt Notation im Mathmodus
  \item[\#]
  \item[\_]
  \item[$\textbackslash$] leitet Befehle ein
  \item[$\sim$] geschütztes Leerzeichen
\end{description}
\item Dass TeX den Mathmodus erwartet \$ \ldots \$ merkt ihr an der Fehlermeldung: "`\texttt{Missing \$ inserted}"' :-)
\end{itemize}

\mode
<article>

\mode
<all>

\end{frame}
\newpage

%%%%%%%%%%%%%%%%%%%%%%%%%%%%%%%%%%%%%%%%%%%%%%%%
\subsection{Codefragmente/Befehle}
%%%%%%%%%%%%%%%%%%%%%%%%%%%%%%%%%%%%%%%%%%%%%%%%
\begin{frame}[fragile]
\frametitle<presentation>{Codefragmente/Befehle}

\mode<article>{
\includethumb
}

\mode
<presentation>
Code auf Slides:
\begin{itemize}
 \item \texttt{block} Umgebung mit eingebettetem \texttt{verbatim}
\end{itemize}

\begin{block}{}
\begin{verbatim}
# blabla -v
\end{verbatim}
\end{block}

\begin{block}{Beispiel mit optionalem Titel}
\begin{verbatim}
# blabla -v
\end{verbatim}
\end{block}

Code im Handout: \texttt{lstlisting}-Umgebung


\mode
<article>

Codefragmente jeder Art, Beispielkonfigs, Befehle etc. werden alle auf den Folien in \texttt{verbatim} gesetzt und in eine \texttt{block} Umgebung eingebettet. Wenn nötig, kann dieser Umgebung auch ein Titel mitgegeben werden:


\begin{block}{}
\begin{verbatim}
# blabla-v
\end{verbatim}
\end{block}

bzw.

\begin{block}{Beispiel mit optionalem Titel}
\begin{verbatim}
# blabla -v
\end{verbatim}
\end{block}

Im Handout keinen Block setzen, hier \texttt{lstlisting} verwenden -- da werden Zeilenumbrüche automatisch gemacht und die Größe automatisch gewählt:
\begin{lstlisting}
code
\end{lstlisting}



\mode
<all>

\end{frame}

\newpage

%%%%%%%%%%%%%%%%%%%%%%%%%%%%%%%%%%%%%%%%%%%%%%%%
\subsection{Listen}
%%%%%%%%%%%%%%%%%%%%%%%%%%%%%%%%%%%%%%%%%%%%%%%%

\begin{frame}[fragile]
\frametitle<presentation>{Listen}


\mode<article>{
\includethumb
}


\mode
<presentation>

\begin{description}
 \item[itemize] Aufzählung gleichrangiger Punkte, Textfolien
\begin{block}{}\footnotesize
\begin{verbatim}
 \begin{itemize}
  \item Listenpunkt
 \end{itemize}\end{verbatim}\normalsize\end{block}
 \item[description] Aufzählung von Punkten mit Erklärung
\begin{block}{}\footnotesize
\begin{verbatim}
 \begin{description}
  \item[Listenpunkt] Erklärung
 \end{description}\end{verbatim}\normalsize\end{block}
 \item[enumerate] Zeitliche/logische Folge, Handlungsanweisung
\begin{block}{}\footnotesize
\begin{verbatim}
 \begin{enumerate}
  \item Listenpunkt
 \end{enumerate}\end{verbatim}\normalsize\end{block}
\end{description}


\mode
<article>

\mode
<all>

\end{frame}
\newpage
%%%%%%%%%%%%%%%%%%%%%%%%%%%%%%%%%%%%%%%%%%%%%%
\subsubsection{itemize}
%%%%%%%%%%%%%%%%%%%%%%%%%%%%%%%%%%%%%%%%%%%%%%
\begin{frame}[fragile]
\frametitle<presentation>{itemize}

\mode<article>{
\includethumb
}

\mode
<presentation>

\begin{itemize}
 \item ich bin ein item erster Ordnung
	\begin{itemize}
	\item und ich zweiter!
 	\item ich auch!!!
	\end{itemize}
 \item ich auch erste Ordnung
\end{itemize}


\mode
<article>

\begin{itemize}
 \item itemize-Listen können geschachtelt werden, man sollte aber nicht mehr als zwei, max. drei Ebenen haben, damit der Seitenspiegel noch vernünftig genutzt ist!
\item itemize-Listen können mit anderen Listen kombiniert werden.
\end{itemize}

Der typische Code einer \verb+itemize+ Umgebung:

\begin{lstlisting}
\begin{itemize}
 \item ich bin ein item erster Ordnung
	\begin{itemize}
	\item und ich zweiter!
 	\item ich auch!!!
	\end{itemize}
 \item ich auch erste Ordnung
\end{itemize}
\end{lstlisting}



\mode
<all>

\end{frame}

\newpage
%%%%%%%%%%%%%%%%%%%%%%%%%%%%%%%%%%%%%%%%%%%%%%
\subsubsection{description}
%%%%%%%%%%%%%%%%%%%%%%%%%%%%%%%%%%%%%%%%%%%%%%
\begin{frame}[fragile]
\frametitle<presentation>{description}

\mode<article>{
\includethumb
}

\mode
<presentation>
\begin{description}
 \item[Begriff] hier kommt die Erklärung
\item[Schachteln] aber klar!
 \begin{description}
	\item[geschachtelt] ... so geht's auch!
 \end{description}
\item[Kombinieren] und kombiniert wird auch:
 \begin{itemize}
  \item Item \#1
  \item Item \#2
 \end{itemize}
 \item[der Tildentrick]~
 \begin{itemize}
  \item Item \#1
  \item Item \#2
 \end{itemize}
\end{description}


\mode
<article>

\begin{itemize}
 \item \verb+description+ immer dann verwenden, wenn ein Begriff näher erklärt werden soll
 \item \textit{NIEMALS} itemize verwenden und dann mit Fake-Auszeichnungen wie Kombinationen aus Fett- oder Kursivdruck oder Bindestrichen eine \verb+description+ vortäuschen.
 \item \verb+description+ ist schachtelbar. Mehr als zwei Ebenen sind aber unübersichtlich!
 \item Kombination aus \verb+description+ und \verb+itemize+ ist möglich; will man sofort nach dem Label einen Zeilenumbruch, dann verwendet man eine \~{} direkt nach dem Label. Danach in der nächsten Zeile die neue Liste anfangen.
\end{itemize}

Der TeX-Code zum Beispiel oben:
\begin{lstlisting}
 \begin{description}
 \item[Begriff] hier kommt die Erklärung
\item[Schachteln] aber klar!
 \begin{description}
	\item[geschachtelt] ... so geht's auch!
 \end{description}
\item[Kombinieren] und kombiniert wird auch:
 \begin{itemize}
  \item Item \#1
  \item Item \#2
 \end{itemize}
 \item[der Tildentrick]~
 \begin{itemize}
  \item Item \#1
  \item Item \#2
 \end{itemize}
\end{description}
\end{lstlisting}



\mode
<all>

\end{frame}
\newpage
%%%%%%%%%%%%%%%%%%%%%%%%%%%%%%%%%%%%%%%%%%%%%%
\subsubsection{enumerate}
%%%%%%%%%%%%%%%%%%%%%%%%%%%%%%%%%%%%%%%%%%%%%%
\begin{frame}[fragile]
\frametitle<presentation>{enumerate}

\mode<article>{
\includethumb
}

\mode
<presentation>
\begin{enumerate}
 \item Item \#1
 \item Item \#2
\begin{enumerate}
 \item Unterpunkt \#1
 \item Unterpunkt \#2
\end{enumerate}
 \item Item \#3
 \begin{itemize}
  \item geschachtelte itemize
 \end{itemize}
 \item Item \#4
 \begin{description}
  \item[geschachtelte description] ...
 \end{description}
\end{enumerate}

\mode
<article>
\begin{itemize}
 \item \verb+enumerate+ für Handlungsaufforderungen und alles andere, wo ein Punkt aus dem anderen folgt und nicht weggelassen werden kann. Das betrifft meist zeitliche oder logische Reihenfolge (\textit{tu dies, dann das ...})
\item Schachtelungen sind möglich; aber nicht übertreiben, damit die Seite optimal genutzt wird
\item Bei einer über mehrere Slides verteilten Liste kann man für eine fortlaufende Zählung sorgen:
\begin{lstlisting}
\begin{enumerate}
        \setcounter{enumi}{6}
\end{lstlisting}
Hier würde die Zählung mit einer "`7"' beginnen, auf dem vorherigen Slide gibt es also schon 6 Punkte.
\end{itemize}

Der Code zum Beispiel:

\begin{lstlisting}
\begin{enumerate}
 \item Item \#1
 \item Item \#2
\begin{enumerate}
 \item Unterpunkt \#1
 \item Unterpunkt \#2
\end{enumerate}
 \item Item \#3
 \begin{itemize}
  \item geschachtelte itemize
 \end{itemize}
 \item Item \#4
 \begin{description}
  \item[geschachtelte description] ...
 \end{description}
\end{enumerate}
\end{lstlisting}

\mode
<all>

\end{frame}
\newpage

%%%%%%%%%%%%%%%%%%%%%%%%%%%%%%%%%%%%%%%%%%%%%%
\subsection{Listen aller Art -- mit Overlays}
%%%%%%%%%%%%%%%%%%%%%%%%%%%%%%%%%%%%%%%%%%%%%%
\begin{frame}[fragile]
\frametitle<presentation>{Listen mit Overlays}

\mode<article>{
\includethumb
}

\mode
<presentation>
\begin{itemize}
 \item Alle Listen, also \texttt{itemize}, \texttt{enumerate} und auch \texttt{description} beherrschen Overlays, i.e. Punkte können nach und nach sichtbar gemacht werden.
\item Diese Option ist nur in Slidedecks aktiv. In Handouts sind Overlays automatisch deaktiviert
 \item statt normaler Listenumgebungen einfach die folgenden Overlay-Umgebungen einsetzen
\small
\begin{block}{}
\begin{center}
\begin{tabularx}{\textwidth}{lX}
\textbf{Normale Umgebung} & \textbf{Overlay-Umgebung} \\
\texttt{itemize} & \texttt{oitemize} \\
\texttt{description} & \texttt{odescription} \\
\texttt{enumerate} & \texttt{oenumerate} \\
\end{tabularx}
\end{center}
\end{block}
\normalsize
\end{itemize}

\mode
<article>

\mode
<all>
\end{frame}
\newpage

\begin{frame}[fragile]
\frametitle<presentation>{Beispiele für Overlays} 

\mode<article>{
\includethumb
}

\mode
<presentation>

\begin{oenumerate}
 \item Ich erscheine zuerst \ldots
 \item \ldots und dann ich
 \item \ldots und dann das Schlußlicht
\end{oenumerate}


\mode
<article>

\mode
<all>

\end{frame}
\newpage

%%%%%%%%%%%%%%%%%%%%%%%%%%%%%%%%%%%%%%%%%%%%%%
\subsection{Tabellen}
%%%%%%%%%%%%%%%%%%%%%%%%%%%%%%%%%%%%%%%%%%%%%%
\begin{frame}[fragile]
\frametitle<presentation>{Tabellen}

\mode<article>{
\includethumb
}

\mode
<presentation>
Verbindliches Tabellenlayout auf Slides:
\begin{block}{\centering{Titel der Tabelle}}
\begin{tabularx}{\textwidth}{XX}
\textbf{Spaltentitel 1} & \textbf{Spaltentitel 2}\\
Zeile 1, Spalte 1 & Zeile 1, Spalte 2\\
Zeile 2, Spalte 1 & Zeile 2, Spalte 2\\
\end{tabularx}
\end{block}

Tabellenlayout im Handout ohne \verb+block+ und mit mehr Linien zur Unterteilung

\mode
<article>
\begin{itemize}
 \item \texttt{tabularx} verwenden! Das erlaubt flexible Spaltenaufteilung, ohne dass man das manuell hinfrickeln muss. Gesamttabellenbreite ist fix, aber die Spaltenbreite kann variabel gewählt werden:
\begin{description}
 \item[l] linksbündig und mit fester Breite (\texttt{r} und \texttt{c} sind auch möglich, sollten aber vermieden werden, damit wir einheitliche Tabellen haben); "`feste Breite"' heißt, dass die Breite der Spalte dem Text angeglichen wird, es erfolgt \textit{kein} automatischer Zeilenumbruch!
 \item[X] linksbündig mit variabler Breite, d.h. nachdem die maximal für die Tabelle mögliche Breite ausgenutzt ist, erfolgt in dieser Spalte automatisch ein Umbruch.
\end{description}
\item Bei der Angabe der Breite: \texttt{textwidth} ist immer die gesamte Textbreite, \texttt{linewidth} wird der Breite angepasst, wenn man sich z.B. in einer \texttt{itemize}-Umgebung befindet
\item Unterschiedliches Layout für Tabellen auf Folien (Beispiel s.o.) und im Handout (siehe unten)
\end{itemize}

\pagebreak

Tabellencode für Folien (Beispiel von oben):

\begin{lstlisting}
\begin{block}{\centering{Titel der Tabelle}}
 \begin{tabularx}{\textwidth}{XX}
  \textbf{Spaltentitel 1} & \textbf{Spaltentitel 2}\\
    Zeile 1, Spalte 1 & Zeile 1, Spalte 2\\
    Zeile 2, Spalte 1 & Zeile 2, Spalte 2\\
 \end{tabularx}
\end{block}
\end{lstlisting}

Tabellenlayout im Handout:

\begin{tabularx}{\textwidth}{|X|X|}
\hline
\textsc{\textbf{Spaltentitel 1}} & \textsc{\textbf{Spaltentitel 2}}\\
\hline
\hline
Zeile 1, Spalte 1 & Zeile 1, Spalte 2\\
Zeile 2, Spalte 1 & Zeile 2, Spalte 2\\
\hline
\end{tabularx}

Dazugehöriger Code:

\begin{lstlisting}
\begin{tabularx}{\textwidth}{|X|X|}
\hline
\textsc{\textbf{Spaltentitel 1}} & \textsc{\textbf{Spaltentitel 2}}\\
\hline
\hline
Zeile 1, Spalte 1 & Zeile 1, Spalte 2\\
Zeile 2, Spalte 1 & Zeile 2, Spalte 2\\
\hline
\end{tabularx}
\end{lstlisting}

\mode
<all>

\end{frame}
\newpage

%%%%%%%%%%%%%%%%%%%%%%%%%%%%%%%%%%%%%%%%%%%%%%%%%%%%%%%%%%%%%%%%%%%%%%%
\begin{frame}[fragile]
\frametitle<presentation>{Tabellen mit Seitenumbruch}

\mode<article>{
\includethumb
}

\mode
<presentation>

\begin{block}{Tabelle mit Seitenumbruch (nur Handout!)}
\footnotesize
\begin{verbatim}
\begin{longtable}{|l|p{9.3cm}|}
\hline
\textsc{\textbf{Titel 1}} & \textsc{\textbf{Titel 2}}\\
\hline
\hline
\endhead
\hline\multicolumn{2}{|r|}{\textit{Fortsetzung ...}}\\
\hline
\endfoot
\endlastfoot
Zeile 1, Spalte 1 & Zeile 1, Spalte 2\\
Zeile 2, Spalte 1 & Zeile 2, Spalte 2\\
\hline
\end{longtable}
\end{verbatim}
\normalsize
\end{block}



\mode
<article>

\begin{itemize}
 \item Für Tabellen mit Seitenumbruch \texttt{longtable} verwenden! Hier muss man etwas rumprobieren, um die passenden Spalten-Breiten zu finden. Folgende Angaben für die Spalten sind möglich:
\begin{description}
 \item[l] linksbündig und mit fester Breite (\texttt{r} und \texttt{c} sind auch möglich, sollten aber vermieden werden, damit wir einheitliche Tabellen haben); "`feste Breite"' heißt, dass die Breite der Spalte dem Text angeglichen wird, es erfolgt \textit{kein} automatischer Zeilenumbruch!
 \item[p\{X.Ycm\}] linksbündig mit angegebener Breite, d.h. Text wird in der entsprechenden Breite umbrochen
\end{description}
\item durch das \texttt{endhead} wird dafür gesorgt, dass der entsprechende "`Header"' auf jeder Seite zu finden ist
\item Definition eines Footers ist sinnvoll, der auf weitere Seiten verweist und dafür sorgt, dass die Tabelle auf jeder Seite unten von einer Linie begrenzt ist; das \texttt{multicolumn} muss natürlich der Zahl der vorhandenen Spalten angeglichen werden (im Beispiel 2)
\item Tabellen mit Seitenumbruch natürlich nur im Handout
\end{itemize}

\pagebreak

Tabellenlayout im Handout:

\begin{longtable}{|l|p{9.3cm}|}
\hline
\textsc{\textbf{Spaltentitel 1}} & \textsc{\textbf{Spaltentitel 2}}\\
\hline
\hline
\endhead
\hline\multicolumn{2}{|r|}{\textit{Fortsetzung auf der nächsten Seite}}\\
\hline
\endfoot
\endlastfoot
Zeile 1, Spalte 1 & Zeile 1, Spalte 2\\
Zeile 2, Spalte 1 & Zeile 2, Spalte 2\\
Zeile 3, Spalte 1 & Zeile 3, Spalte 2\\
Zeile 4, Spalte 1 & Zeile 4, Spalte 2\\
Zeile 5, Spalte 1 & Zeile 5, Spalte 2\\
Zeile 6, Spalte 1 & Zeile 6, Spalte 2\\
Zeile 7, Spalte 1 & Zeile 7, Spalte 2\\
Zeile 8, Spalte 1 & Zeile 8, Spalte 2\\
Zeile 9, Spalte 1 & Zeile 9, Spalte 2\\
Zeile 10, Spalte 1 & Zeile 10, Spalte 2\\
Zeile 11, Spalte 1 & Zeile 11, Spalte 2\\
Zeile 12, Spalte 1 & Zeile 12, Spalte 2\\
Zeile 13, Spalte 1 & Zeile 13, Spalte 2\\
Zeile 14, Spalte 1 & Zeile 14, Spalte 2\\
Zeile 15, Spalte 1 & Zeile 15, Spalte 2\\
Zeile 16, Spalte 1 & Zeile 16, Spalte 2\\
Zeile 17, Spalte 1 & Zeile 17, Spalte 2\\
Zeile 18, Spalte 1 & Zeile 18, Spalte 2\\
Zeile 19, Spalte 1 & Zeile 19, Spalte 2\\
Zeile 20, Spalte 1 & Zeile 20, Spalte 2\\
Zeile 21, Spalte 1 & Zeile 21, Spalte 2\\
Zeile 22, Spalte 1 & Zeile 22, Spalte 2\\
Zeile 23, Spalte 1 & Zeile 23, Spalte 2\\
Zeile 24, Spalte 1 & Zeile 24, Spalte 2\\
Zeile 25, Spalte 1 & Zeile 25, Spalte 2\\
Zeile 26, Spalte 1 & Zeile 26, Spalte 2\\
Zeile 27, Spalte 1 & Zeile 27, Spalte 2\\
Zeile 28, Spalte 1 & Zeile 28, Spalte 2\\
Zeile 29, Spalte 1 & Zeile 29, Spalte 2\\
Zeile 30, Spalte 1 & Zeile 30, Spalte 2\\
Zeile 31, Spalte 1 & Zeile 31, Spalte 2\\
Zeile 32, Spalte 1 & Zeile 32, Spalte 2\\
Zeile 33, Spalte 1 & Zeile 33, Spalte 2\\
Zeile 34, Spalte 1 & Zeile 34, Spalte 2\\
Zeile 35, Spalte 1 & Zeile 35, Spalte 2\\
Zeile 36, Spalte 1 & Zeile 36, Spalte 2\\
Zeile 37, Spalte 1 & Zeile 37, Spalte 2\\
Zeile 38, Spalte 1 & Zeile 38, Spalte 2\\
Zeile 39, Spalte 1 & Zeile 39, Spalte 2\\
Zeile 40, Spalte 1 & Zeile 40, Spalte 2\\
Zeile 41, Spalte 1 & Zeile 41, Spalte 2\\
Zeile 42, Spalte 1 & Zeile 42, Spalte 2\\
Zeile 43, Spalte 1 & Zeile 43, Spalte 2\\
Zeile 44, Spalte 1 & Zeile 44, Spalte 2\\
Zeile 45, Spalte 1 & Zeile 45, Spalte 2\\
Zeile 46, Spalte 1 & Zeile 46, Spalte 2\\
Zeile 47, Spalte 1 & Zeile 47, Spalte 2\\
Zeile 48, Spalte 1 & Zeile 48, Spalte 2\\
Zeile 49, Spalte 1 & Zeile 49, Spalte 2\\
Zeile 50, Spalte 1 & Zeile 50, Spalte 2\\
Zeile 51, Spalte 1 & Zeile 51, Spalte 2\\
Zeile 52, Spalte 1 & Zeile 52, Spalte 2\\
\hline
\end{longtable}

Dazugehöriger Code:

\begin{lstlisting}
\begin{longtable}{|l|p{9.3cm}|}
\hline
\textsc{\textbf{Spaltentitel 1}} & \textsc{\textbf{Spaltentitel 2}}\\
\hline
\hline
\endhead
\hline\multicolumn{2}{|r|}{\textit{Fortsetzung auf der nächsten Seite}}\\
\hline
\endfoot
\endlastfoot
Zeile 1, Spalte 1 & Zeile 1, Spalte 2\\
Zeile 2, Spalte 1 & Zeile 2, Spalte 2\\
...
\hline
\end{longtable}
\end{lstlisting}

\mode
<all>

\end{frame}
\newpage

%%%%%%%%%%%%%%%%%%%%%%%%%%%%%%%%%%%%%%%%%%%%%%
\subsection{Warnhinweise}
%%%%%%%%%%%%%%%%%%%%%%%%%%%%%%%%%%%%%%%%%%%%%%
\begin{frame}[fragile]
\frametitle<presentation>{Warnhinweise und Co.}


\mode<article>{
\includethumb
}


\mode
<presentation>
\begin{description}
 \item[Achtung/Warning] Die allerhöchste Warnstufe, die wir haben. Hier sind konkret Daten oder Hardware bedroht, wenn man den Hinweis nicht befolgen würde.
 \item[Wichtig/Important] Zweithöchste Warnstufe. Wichtig, aber Nichtbeachten hat keine (negativen) Folgen.
 \item[Tipp/Tip] Interessante Zusatzinformation.
 \item[Hinweis/Note] Ähnlich wie Tipp. Typischer Einsatz, wenn man auf Versionsunterschiede o.ä. hinweisen möchte.
\end{description}

Format:
\begin{block}{Warning: Cats Are in Danger When Microwaved}
 blablabla
\end{block}

\mode
<article>

Wir unterscheiden in den Unterlagen zwischen vier Typen von Warnhinweisen:

\begin{description}
 \item[Achtung/Warning] Allerhöchste Warnstufe. Konkrete Bedrohung wie Datenverlust oder Beschädigung der Hardware. Alles andere ist keine Warnung, sondern sollte als \textit{Wichtig} gehandelt werden.
 \item[Wichtig/Important] Zweithöchste Warnstufe. Wichtige Information für den Benutzer, die für sein Setup wichtig ist, aber deren Nichtbeachten keinerlei negative Folgen für ihn hat.
 \item[Tipp/Tip] Interessante Zusatzinformation für den Leser. Sie ist nicht notwendig, macht ihn aber schlauer.
 \item[Hinweis/Note] Siehe \textit{Tipp}. Einsatzgebiet sind zum Beispiel Änderungen in einer Software von Version zu Version oder geänderte Tastaturshortcuts -- einfach Fakten, die der Leser eigentlich schon hat, die sich aber geändert haben und deshalb rausgestellt werden müssen.
\end{description}

Bei \textit{Tipp} und \textit{Hinweis} gibt es keine Gefahr, also beschränkt euch auf eine Kurzdarstellung der Info, die ihr vermitteln wollt. Überschriften müssen diese Boxen aber trotzdem haben.

Beispielcode (presentation):
\begin{lstlisting}
\begin{block}{Warning: Cats Are in Danger When Microwaved}
blablabla
\end{block}
\end{lstlisting}

Für die Boxen im Handout haben wir selbstdefinierte Makros namens \texttt{b1warning}, \texttt{b1important}, \texttt{b1tip} und \texttt{b1note}.

Beispielcode für einen Warnhinweis:
\begin{lstlisting}
\begin{b1warning}{Cats Are in Danger When Microwaved}
blabla
\end{b1warning}
\end{lstlisting}
 
Alle diese Makros funktionieren nur innerhalb einer geklammerten \texttt{article} Umgebung. Wenn die nicht verwendet werden kann, weil zum Beispiel Listings im gleichen Abschnitt verwendet werden, initialisiert sie einfach neu. Die entsprechende Fehlermeldung bemängelt in solchen Fällen:

\begin{lstlisting}
Environment bclogo undefined 
\end{lstlisting}


\mode<article>{

\begin{b1warning}{Cats Are in Danger When Microwaved}
 blabla
\end{b1warning}

Die Signalworte expandiert das Makro selbst und die werden auch sprachabhängig richtig gesetzt. 

\begin{b1important}{Cats Do Not Like Being Microwaved}
blablabla
\end{b1important}

Denkt dran, die Hinweise knapp, aber vollständig zu halten und niemals Listings (\texttt{verbatim}, \texttt{lstlisting}) oder Grafiken innerhalb der Umgebung zu verwenden.

\begin{b1tip}{Cats Like Trees}
blablablabla
\end{b1tip}

In den Slides verwendet weiterhin das \texttt{block} Environment wie gehabt. Die Formatierung ist nicht zu schlagen ;)

\begin{b1note}{More Information For Cat Lovers}
 blabla
\end{b1note}

% \begin{bclogo}[logo=\bctrefle,noborder=true,epBarre=1]{}
% Alle diese Makros funktionieren nur innerhalb einer geklammerten \texttt{article} Umgebung. Wenn die nicht verwendet werden kann, weil zum Beispiel Listings im gleichen Abschnitt verwendet werden, initialisiert sie einfach neu. Die entsprechende Fehlermeldung bemängelt in solchen Fällen:
% \end{bclogo}
% 
% \begin{bclogo}[logo=\bctrefle,noborder=true,barre=none]{}
% Alle diese Makros funktionieren nur innerhalb einer geklammerten \texttt{article} Umgebung. Wenn die nicht verwendet werden kann, weil zum Beispiel Listings im gleichen Abschnitt verwendet werden, initialisiert sie einfach neu. Die entsprechende Fehlermeldung bemängelt in solchen Fällen:
% \end{bclogo}
}





\mode
<all>

\end{frame}
\newpage

%%%%%%%%%%%%%%%%%%%%%%%%%%%%%%%%%%%%%%%%%%%%%%
\subsection{Bilder und Grafiken}
%%%%%%%%%%%%%%%%%%%%%%%%%%%%%%%%%%%%%%%%%%%%%%
\begin{frame}[fragile]
\frametitle<presentation>{Bilder und Grafiken}

\mode<article>{
\includethumb
}

\mode
<presentation>
\begin{figure}[htb!]
        \begin{center}
        \includegraphics[height=0.2\textheight]{common/artwork/b1-tux}
        \end{center}
        \caption{B1 Tux}
        \label{fig:tux}
\end{figure}
\begin{block}{}\small
\begin{verbatim}
\begin{figure}[htb!]
 \begin{center}
   \includegraphics[height=0.2\textheight]{common/bild}
 \end{center}
 \caption{B1 Tux}
 \label{fig:tux}
\end{figure}
\end{verbatim}\normalsize\end{block}

\mode
<article>

Grafiken werden im SVN entweder als SVG abgelegt, falls sie selbst erstellt wurden, oder bei Screenshots als PNG.

Das Einbinden erfolgt wie im Beispiel oben mit der \texttt{figure}-Umgebung. Bei Angabe des Bild-Namens kann die Endung weggelassen werden.

Die Skalierung der Grafik müsst ihr einfach testen -- die Angabe der Größe kann über \texttt{height} oder \texttt{width} erfolgen und wird am einfachsten als Faktor von \texttt{textheight} oder \texttt{textwidth} angegeben. Natürlich könnt ihr auch direkt eine feste Größe angeben. Achtet bitte darauf, dass die Bild-Unterschrift zu sehen ist und das Bild nicht in die Folienbeschriftung "`hineinragt"'.

Bitte gebt mit \texttt{caption} eine vernünftige Bild-Unterschrift an.

Möchtet ihr später auf die Abbildung verweisen können, dann fügt \emph{unter} dem \texttt{caption}-Eintrag ein \texttt{label} ein. Benennt das Label sinnvollerweise mit \texttt{fig:mein-label}.

\mode
<all>

\end{frame}
\newpage
