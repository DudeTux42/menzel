\section{Typographische Konventionen}

\begin{frame}[fragile]
\frametitle<presentation>{Typographische Konventionen}


\mode<article>{
\includethumb
}


\mode
<presentation>

        \begin{center}
        \huge{Typographische Konventionen}
        \end{center}

\mode
<article>

Auf den folgenden Seiten werden kurz die in dieser Schulungsunterlage angewandten typographischen Konventionen vorgestellt.

\mode
<all>

\end{frame}
\newpage

\begin{frame}[fragile]
\frametitle<presentation>{Typographische Konventionen}

\mode<article>{
\includethumb
}

\mode
<presentation>
\small
\begin{block}{\centering{Typographische Konventionen}}
\begin{center}
\begin{tabularx}{\textwidth}{lX}
\textbf{Element} & \textbf{Auszeichnung}\\
Datei- und Verzeichnisname & \texttt{/etc/passwd}\\
Benutzername & \texttt{tux} \\
URL & \url{http://www.b1-systems.de} \\
Variable & \texttt{PATH} \\
Platzhalter/Variable & \texttt{\textit{dateiname}} oder \texttt{<dateiname>} \\
Kommando & \texttt{ls -l} \\
Menüpunkt/Eingabefeld/Button & \textit{OK} \\
Taste bzw. Tastenkombination & \textit{Alt+F1} \\
Hervorhebung & \emph{Wichtige Information} \\
\end{tabularx}
\end{center}
\end{block}
\normalsize


\mode
<article>

Die folgende Liste führt die wichtigsten typographischen Elemente auf, die in dieser Schulungsunterlage verwendet werden:

\begin{description}
\item[\texttt{/etc/passwd}] Datei- und Verzeichnisnamen werden als \texttt{/pfad/zur/datei} gekennzeichnet.

\item[\texttt{tux}] Benutzernamen werden genau so wie Datei- und Verzeichnisnamen hervorgehoben: \texttt{tux}.

\item[\url{http://www.b1-systems.de}] URL.

\item[\texttt{PATH}] Angabe einer Variable mit \texttt{PATH}.

\item[\texttt{\textit{dateiname}} | \texttt{<dateiname>}] Ein Platzhalter ist entweder durch Kursivschrift\newline
  (\texttt{PATH=\textit{dateiname}}) oder durch den Einschluss in spitze Klammern\newline
  (\texttt{PATH=<dateiname>}) gekennzeichnet.

\item[\texttt{ls -l}] Kommandos werden wie Datei- und Verzeichnisnamen und Benutzernamen hervorgehoben.

\item[\textit{OK}] Menüpunkt/Eingabefeld/Button.

\item[\textit{Alt+F1}] Tastatureingabe bzw. Tastenkombination.

\item[\emph{Wichtige Information}] Hervorhebung wichtiger Informationen.
\end{description}


\mode
<all>

\end{frame}
\newpage

\begin{frame}[fragile]
\frametitle<presentation>{Shell-Prompt}

\mode<article>{
\includethumb
}

\mode
<presentation>

Unterschiedliche Prompts für Superuser \texttt{root} und "`gewöhnliche"' Benutzer:

\begin{block}{Aufruf eines Befehls als "`gewöhnlicher"' Benutzer}
\begin{verbatim}
$ rpm -qi bash  
\end{verbatim}
  
\end{block}

\begin{block}{Aufruf eines Befehls als \texttt{root}}
\begin{verbatim}
# rpm -Uvh bash-<version>.rpm
\end{verbatim}
\end{block}

\mode
<article>

Zur Kennzeichnung, ob ein Kommando von einem "`gewöhnlichen"' Benutzer ausgeführt werden kann, oder ob nur der Superuser \texttt{root} dies darf, werden unterschiedliche Prompts in der Shell verwendet:

\begin{itemize}

\item Aufruf eines Befehls als "`gewöhnlicher"' Benutzer:
\begin{lstlisting}
$ rpm -qi bash
\end{lstlisting}
Das Zeichen \texttt{\$} ist der Prompt für einen "`gewöhnlichen"' Benutzer.

\item Aufruf eines Befehls als \texttt{root}:
\begin{lstlisting}
# rpm -Uvh bash-<version>.rpm
\end{lstlisting}
Das Zeichen \texttt{\#} steht für den Prompt von \texttt{root}.

\end{itemize}


\mode
<all>

\end{frame}
\newpage
\cleardoublepage
