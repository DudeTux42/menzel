%\def\pgfsysdriver{pgfsys-pdftex.def}
\documentclass[11pt,handout]{beamer}
\beamersetuncovermixins{\opaqueness<1>{25}}{\opaqueness<2->{15}}

\usepackage[T1]{fontenc}
\usepackage[utf8]{inputenc}
\usepackage{listings}
\lstset{
  breaklines=true,
  basicstyle=\small\ttfamily,
  extendedchars=true,
  inputencoding=utf8,
  frame=none,
  literate={Ö}{{\"O}}1 {Ä}{{\"A}}1 {Ü}{{\"U}}1 {ß}{{\ss}}2 {ü}{{\"u}}1 {ä}{{\"a}}1 {ö}{{\"o}}1 {µ}{\textmu}1
}
\usepackage[\configLang]{babel}
\usepackage{graphicx}
\usepackage{color}
\usepackage{hyperref}
\usepackage{multirow}
\usepackage{multicol}
\usepackage{tabularx}
\usepackage{longtable,makecell}
\usepackage{pgf}
\usepackage{pgfpages}
\usepackage[gen]{eurosym}
\usepackage{xspace}
% Paket für Profiling
\RequirePackage{versions}

%\setjobnamebeamerversion{lvm}
\usetheme{B1CustomNew}
%\usepackage{beamerthemeshadow}


\definecolor{b1blau}{RGB}{34,71,121}
\definecolor{b1orange}{RGB}{252,189,43}
\definecolor{grau}{RGB}{40,40,40}
\definecolor{greylist}{RGB}{224,224,224}
\definecolor{tmagenta}{RGB}{226,0,116}
%% ugly workaround for acroread screwing up the colors
\pdfpageattr {/Group << /S /Transparency /I true /CS /DeviceRGB>>}

\setbeamercolor{structure}{fg=b1blau}
\setbeamercolor*{palette primary}{bg=b1blau}
\setbeamercolor*{palette secondary}{bg=b1blau}
\setbeamercolor*{palette tertiary}{bg=b1blau}

\setbeamercolor{palette primary}{use=structure,fg=white,bg=b1blau!100!black}
\setbeamercolor{palette secondary}{use=structure,fg=white,bg=b1blau!100!white}
\setbeamercolor{palette tertiary}{use=structure,fg=white,bg=b1blau!100!white}
\setbeamercolor{palette quaternary}{use=structure,fg=white,bg=b1blau!100!white}
% 
\setcounter{section}{0}

